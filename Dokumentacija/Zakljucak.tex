\chapter{Zaključak i budući rad}

		Rad na ovom projektu u sklopu predmeta Programsko inženjerstvo pružio nam je bogato iskustvo, kako u organizaciji rada u timu, tako i u upotrebi brojnih alata koje smo koristili te samoj izradi web aplikacije. 
		Naš je zadatak bio u nekoliko mjeseci napraviti web aplikaciju kampa \textit{Mlade nade} namijenjenu organizatorima, kako bi im olakšali posao organizacije, ali i budućim sudionicima i animatorima, koji na ovaj način puno brže i lakše podnose prijave.\newline
		\indent Na samom početku prvog ciklusa bilo je potrebno opisati i razraditi projektni zadatak. Iako smo u ranoj fazi mislili da bi lakše bilo opisivati funkcionalnosti usputno, za vrijeme rada na aplikaciji, pokazalo se da je detaljna razrada funkcionalnosti prije same izrade bila ne samo korisna, nego i nužna za jednostavniju organizaciju i rad. Dijagrami i obrasci uporabe izrađeni u prvom ciklusu (dijagrami obrazaca uporabe, sekvencijski dijagrami, dijagrami razreda i baze podataka) bili su temelj za daljnji rad članova tima zaduženih za backend i frontend. Dobro izrađena dokumentacija pomogla nam je izbjeći nedoumice oko razvoja funkcionalnosti i olakšala raspodjelu posla među članovima tima. U prvoj fazi projekta neki od nas su se prvi puta susreli s korištenim alatima i programskim jezicima odabranima za rad. Savladali smo osnove korištenja sustava Git, LaTex i izradu dokumentacijskih dijagrama te prvi puta u praksi koristili bazu podataka.\newline
		\indent U drugoj fazi projekta naišli smo na veće izazove. Kako većina nas nije imala iskustva u radu s korištenim alatima, bili smo primorani samostalno učiti kako bismo ostvarili zadani cilj. Bilo je potrebno dokumentirati ostale UML dijagrame i napisati popratnu dokumentaciju kako bi korisnici mogli lakše koristiti našu aplikaciju i kako bismo olakšali rad na unapređenju iste.\newline
		\indent Naša komunikacija ostvarena je čestim sastancima i zajedničkim radom što se pokazalo vrlo efikasnim. Uvidjeli smo da je za rad u timu sastavljenom od ovako velikog broja članova nužna dobra komunikacija i informiranost svih članova grupe o napretku projekta. Shvatili smo važnost dobre organizacije u smislu raspodjele posla među članovima tima te vremena potrebnog da se zadaci izvrše. Da bi tim funkcionirao kao cjelina, jako je važno da svaki član tima odradi svoj dio posla savjesno i na vrijeme, kako bi se izbjegao nepotrebni angažman ostalih članova tima. Sva stečena iskustva i znanja koristit ćemo i nadograđivati u budućim projektima.
		
		\eject 